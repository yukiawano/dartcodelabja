\ochapter{Step 3: チャットアプリを使って,基本的なDart言語の特性を学ぶ}

Dart言語は,広範囲にわたる開発者になじみがあります.これは,単一継承,中括弧,セミコロンからなるクラスベースのオブジェクト指向言語です.構文はすぐに認識でき,そしてその概念は簡単にできることです.

あなたがチャットコードを見て回ることで,あなたはDart言語の基礎を学びます. あなたはDartエディタのコードナビゲーション機能を使うことができ,検索,アウトラインそして不慣れなコードをより迅速に高速化するために定義へのジャンプします.

\section{目的}

\begin{enumerate}
\item Dartエディタを使用してコードを検索します
\item クラスやメソッドの定義にジャンプします
\item ファイル構造を表示するためにアウトラインビューを使用します
\item クラス,スーパークラス,総称型,関数,メソッド,変数,ライブラリ等について学習します
\item 任意の静的な型について学びます
\item サンプルアプリの基本的なレイアウトを理解します
\item Dartエディタ内の警告を理解します
\item Dartエディタ内のエラーを理解します
\end{enumerate}

\section{あなたがコードを始める前に}

{\bf dartchatのstart-here ディレクトリからコーディングの旅を開始します.}あなたが追いつく必要がある場合や,またはやり直す必要があるなら,このコードラボのこの部分に対して,start-hereにstep03をコピーすることができます.

\section{ウォークスルー}

Dart言語やエディタ機能をデモするために,HTMLページ上でチャットウィンドウを表すクラスを作成します. このプロセスでは,DartエディタやDart言語についての詳細をお教えします.

\subsection{プロジェクトレイアウト}

コーディングを開始する前に,プロジェクトのレイアウトを理解しておくと役立ちます
.Dartエディタでdartchatプロジェクトの内部で,finishedディレクトリを開きます.今,clientディレクトリを完成アプリの完全なレイアウトを得るために開きます.

\imgw{step3/img_30.jpg}

あなたは,clientディレクトリでchat-client.dart.に取り組み,多くの時間を費やします.そのプロジェクトは,クライアント(chat-client.dart)とサーバ(chatserver.dart)の両方のコードが含まれています.

\subsection{コードの検索}

``Very first edit'' (著者があなたのために含んだ,参考コメントです:) )
を見つけるためにツールの右上にある,エディタの検索機能を使用して,
そして編集を開始できます.

\img{step3/img_31.jpg}

検索結果はエディタの下部に専用のビューに表示されます.

\img{step3/img_32.jpg}

ファイルを開いていくつかのDartを書き始めるのに,start-hereからchat-client.dartをダブルクリックしください!

\hint{あなたの編集の全てにおいてstart-hereを使用します. もしやり直したいときのために,フォールバックすることを目的としたstep03を編集しないようにしてください.}

\subsection{トップ・レベルの変数を追加します.}

client/chat-client.dartが開いて,先頭へスクロールしたこと確認します.チャットウィンドウを表すオブジェクトにトップレベルの変数を追加します.変数名としてchatWindowそして型アノテーションとしてChatWindowを使います.

\begin{verbatim}
// client/chat-client.dart
...
UsernameInput usernameInput;
// Step 3: Very first edit
// Step 3: Add variable for chat window.
ChatWindow chatWindow;
class ChatConnection {
...
\end{verbatim}

\hint{一つのクラス内に,全てを含む代わりにトップ・レベル関数や変数を使用することができます.}

ファイルを保存します.型アノテーションを使用して新しい変数を追加した後,エディタはあなたにそれがChatWindowが何であるかを知らないことを示す警告(黄色下線で)提供していることに注意してください.我々はまだそれを定義していないので,それは理にかなっています.

\img{step3/img_33.jpg}

\advancedtopic{Dartはオプショナルな型付の言語です.つまり,必要な場合には型を使い,必要ない場合には型を使う必要がないという事です.もしランタイムがあなたが参照している型を見つけることができない場合,その静的な型アノテーションを無視して,ダイナミック型(Dynamic)の変数として扱います.(Dynamic型とは型アノテーションが何も提供されなかった場合の代役の型です.)型アノテーションで示されたが不足している型があったとしても,それはプログラムの実行やコンパイルを停止させません.(なぜなら,もう一度思い出していただきたいのですが,Dartはオプショナルな型付の言語だからです.)EditorがChatWindowを見つけることができなかったとしても,Editorは警告を発するだけです.}

\subsection{オブジェクトのインスタンスを作成}

ChatWindowの新しいインタンスを作るために,main()までスクロールダウンしてください.多くの他の言語と同様に,新しいオブジェクトを構築するための新しいキーワードを使用します.

\begin{verbatim}
// client/chat-client.dart
main() {
  // Step 4: Identify elements by ID.
  TextAreaElement chatElem = null;
  InputElement usernameElem = null;
  InputElement messageElem = null;
  // Step 3: Instantiate ChatWindow.
  chatWindow = new ChatWindow(chatElem);
  usernameInput = new UsernameInput(usernameElem);
  ...
\end{verbatim}

エディタが赤の下線とともにエラーを表示していることを確認してください.具体的には,''型ChatWindowが見つかりません''(no such type''ChatWindow'')と表示されているはずです.

\img{step3/img_34.jpg}

エラーは,コンパイルすることや実行することからプログラムを停止しますので,これは継続する前に修正しなければならない問題です.Dartは,実際のエラーの状況では,その結果の数を最小限に抑えようとしますが,Dartでは構築したいオブジェクトの種類を知る必要がありません.

\subsection{クラスを定義}

エラーを消すには,ChatWindowクラスを追加します.''Define the ChatWindow class''コメントを検索し,次のクラスを追加します.

\begin{verbatim}
// client/chat-client.dart
...
// Step 3: Define the ChatWindow class.
class ChatWindow extends View<TextAreaElement> {
}
...
\end{verbatim}

エディタは不明なコンストラクタについて,あなたにエラーを出します.心配しないで,我々は次でこれを埋めるだろう.ChatWindowクラスは(既にアプリケーションに正しく定義されている)Viewクラスを継承します.Viewクラスはさらにビューがカプセル化されたHTML要素の型を指定する総称型を使用しています.上記のコードは,"ChatWindowがTextAreaElementの特殊なビューである"と言っています.

\advancedtopic{Dartは一般にジェネリック型\footnote{訳者追加資料: ジェネリックプログラミング(日本語Wikipediaより) - \url{http://ja.wikipedia.org/wiki/ジェネリックプログラミング}}として知られているパラメータ化された型をサポートしています.Dartはオプショナルな型付の言語なので,ジェネリック型を使う必要はありません.しかしながら,それらはコードの表現力\footnote{読みやすさ}をより良くするためにとても役に立つツールです.Dartのジェネリック\footnote{\url{http://www.dartlang.org/docs/language-tour/#generics
}}は他の主流な言語のものに比べていくらかシンプルだと思います.}

Viewのコンストラクタに委譲するコンストラクタをChatWindowに追加します.

\begin{verbatim}
// client/chat-client.dart
...
// Step 3: Define the ChatWindow class.
class ChatWindow extends View<TextAreaElement> {
ChatWindow(TextAreaElement elem) : super(elem);
...
\end{verbatim}

:の次に来るものはクラスの初期化リスト(Initializer list)\footnote{\url{http://www.dartlang.org/docs/language-tour/#initializer-list}}です.初期化リストはfinalの変数を初期化したり,スーパークラスのコンストラクタを呼ぶ\footnote{訳者注: Finalに指定されている変数はDartではオブジェクトがthisに割り当てられる前に初期化される必要があります.そのため,コンストラクタの本体部分が実行される前に,初期化リストを使ってfinalの変数を初期化します.}のに使われます.

次にChatWindowに2つのパブリックメソッドと1つのプライベートメソッドを追加します.これらのメソッドは\verb|<textarea>|要素にメッセージを表示します.

\begin{verbatim}
// client/chat-client.dart
...
class ChatWindow extends View<TextAreaElement> {
  ChatWindow(TextAreaElement elem) : super(elem);
  displayMessage(String msg, String from) {
    _display("$from: $msg\n");
  }
  displayNotice(String notice) {
    _display("[system]: $notice\n");
  }
  _display(String str) {
    elem.text = "${elem.text}$str";
  }
}
...
\end{verbatim}

上記のコードでは,\_display()メソッドはライブラリプライベート(Library Private)です.アンダースコア(\_)が先頭についている名前(のメソッド)はそれが定義されているライブラリ内においてプライベートです.ChatWindowクラスは,\#library('chat-client')とかかれたファイルに定義されており,その結果\_display()は,'chat-client'ライブラリに属するコードのみから見えます.他の2つのメソッドはパブリックです.

elemのtextプロパティは\verb|<textarea>|タグの内容(Content)にアクセスしています.Dartの便利な機能の一つは,文字列の内挿(挿入,差し込み,Interpolation)です.上の3つのメソッドのすべてで使われています.あなたは\$接頭辞を変数につけることで変数を直接参照することができて,''\$from: \$msg$\backslash$n''のように文字列を作成することができます.\_display()で使用されているelem(の定義)はスーパークラスであるViewを見ると見つけることができます.ChatWindowはViewを継承(Extend)したものなので,Viewのインスタンス変数へアクセスすることができます.

\subsection{コードのアウトラインを見てください}

ChatWindowクラスに新しく追加されており,あなたはエディタを使うことによって簡単にchat-client.dartファイルのアウトラインをブラウズ可能です
chat-client.dartファイルを選択して,メニューから\verb|Tools > Outline|を選択します.

\img{step3/img_36.jpg}

chat-client.dartで見つかったクラスやメソッド,変数がアウトラインビューには今表示されていることかと思います.

\img{step3/img_37.jpg}

\section{上級編}

strings\footnote{\url{http://www.dartlang.org/docs/language-tour/#strings}}やclasses\footnote{\url{http://www.dartlang.org/docs/language-tour/#classes}}などを見てDart言語について学んでください.

クラス名を右クリックし,Open Declarationを選択します.これをchat-client.dartで定義されているクラスやDartに予め定義されているクラスなどで試してみてください.

\img{step3/img_38.jpg}

メソッドを右クリックして,Open Callersを選択してください.このメソッドを呼び出しているすべての場所の一覧を見ることができると思います.

\img{step3/img_39.jpg}

\clearpage
