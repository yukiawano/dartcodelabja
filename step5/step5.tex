\section{Step 5: JSONを利用してデータをエンコード・デコードする}

JSON(Javascript Object Notation)は配列や数値,文字列,論理値,そしてマップなどの構造があるデータをエンコードするのに最適なテキストフォーマットです.システム全体で利用するデータを簡単に扱えるようJSONはプログラミング言語やライブラリにわたって,強力にサポートされています.

ここで様々なデーターオブジェクトがJSON文字列にどのようにエンコードされるのかを見ておきましょう:

\begin{verbatim}
// A Dart map
var data = {'scores': [12,54,99]};
assert(data is Map);
assert(data['scores'] is List);
assert(data['scores'][0] == 12);
// Convert to JSON string
var json = JSON.stringify(data);
// A Dart map encoded as a JSON string
assert(json == '{"scores": [12, 54, 99]}');
\end{verbatim}

Dartのマップやリスト表現にとても似ていることが分かります.

\subsection{目的}

\begin{enumerate}
\item JSONについて学ぶ
\item オブジェクトをJSON文字列へエンコードする
\item JSON文字列をオブジェクトにデコードする
\end{enumerate}

\subsection{コード}

もし,作業につまづいた場合はstep05のフォルダの内容をstart-hereフォルダにコピーすることで,この部分の作業内容を終わらせた状態にすることができます.

\subsection{ウォークスルー}

$ client/chat-client.dart $を開き,ファイルの先頭へスクロールしてください.JSONライブラリを追加し,JSONをエンコード・デコードしましょう.そして,Dart Editorの新しいいくつかの機能も試してみましょう.

\subsubsection{dart:jsonをインポートする}

Dartに付属されているJSONの機能を利用する場合,dart:jsonを$ client/chat-client.dart $にあるchat-client libraryへインポートする必要があります.

\begin{verbatim}
// client/chat-client.dart
#library('chat-client');
#import('dart:html');
// Step 5: Import the JSON library.
#import('dart:json');
ChatConnection chatConnection;
\end{verbatim}

\subsubsection{送信のためにメッセージをエンコードする}

ChatConnectionの中でsend()を見つけ,ユーザー名とメッセージの両方を一つのJSON文字列へエンコード(``stringify'')するためのコードを追加しましょう.

\begin{verbatim}
// client/chat-client.dart
send(String from, String message) {
// Step 5. Encode from and message into one JSON string.
var encoded = JSON.stringify({'f': from, 'm': message});
_sendEncodedMessage(encoded);
}
\end{verbatim}

上野コードでは\{ \}の文法を利用してマップ表現を作成しています.''このメッセージはこのユーザーから送られました''ということを的確に伝えられるよう,ユーザー名とメッセージの両方はひとつのマップにまとめられます.そして,そのマップはJSON.stringify()を利用してJSON文字列にエンコードされます.

一旦メッセージが文字列へエンコードされたら,それはWebSocketを通じて送信するために\_sendEncodedMessage()へ渡されます.(この部分は次のステップでコードを書きます.)

\subsubsection{受信したメッセージをデコードする}

JSON.parse()を利用して,JSON文字列をDartオブジェクトへデコードします.ChatConnectionの中の\_receivedEncodedMessage()を見つけて,メッセージをデコードしてチャットウインドウに表示するコードを追加します.

\begin{verbatim}
// client/chat-client.dart
_receivedEncodedMessage(String encodedMessage) {
// Step 5: Decode a JSON string and display it in the chat window.
Map message = JSON.parse(encodedMessage);
if (message['f'] != null) {
chatWindow.displayMessage(message['m'], message['f']);
}
}
\end{verbatim}

メッセージに送信者のユーザー名がつけられていることを確認するためのエラーチェックのためのコードを含んでいます.(``f''は''から(from)''の短縮形です.)

\subsection{上級編}

\begin{enumerate}
\item 不正なフォーマットのメッセージの取り扱いをしてください.クライアントがパースできないメッセージや理解できないメッセージを受信した場合に,役に立つメッセージを表示してください.
\end{enumerate}
