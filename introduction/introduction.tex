\chapter*{イントロダクション}
\addcontentsline{toc}{chapter}{イントロダクション}

このCodelabではDartを利用して簡単なチャットアプリケーションを構築して,実行していきます.このCodelabの中では,次のことを学びます.

\begin{itemize}
\item Dartプログラミング言語の基礎
\item Dart HTMLライブラリの使い方
\item WebSocketsを使った双方向コミュニケーション
\item Dartエディタの基礎
\item オンラインリソースからの答えの見つけ方
\end{itemize}

\section{事前条件}

このCodelabではWebプログラミングの基礎について知識があることを前提としています.HTML,CSS,そしてJavascriptについて知っている必要があります.Java,C\#,Javascriptなどといったプログラミング言語への経験も持っているとより良いです.

このCodelabでは,次のVideoチュートリアルを見ていることを前提としています.\footnote{訳者注:でも色々と知っておかないといけないと気負いをする必要はありません.Codelabはステップ・バイ・ステップなので簡単についていけると思います.}

\begin{itemize}
\item Google I/O 101: Introduction to Dart with Seth Ladd\footnote{\url{http://www.youtube.com/watch?v=vT1KmTQ-1Os}}
\item Google I/O 101: Dart Editor with Devon Carew\footnote{\url{http://www.youtube.com/watch?v=9PHMKzgrmxE}}
\end{itemize}

\section{インストール}

このCodelabではDartエディタを利用します.Mac, Windows, Linux向けのエディタのビルドは\url{http://dartlang.org/editor/}で見つけることができます.

\section{追加資料}

このCodelabはステップ・バイ・ステップのインストラクションでできていて,簡単についていくことができます.しかし,次のオンラインリソースを読み込んでおいて,アクセスできるようにしておくととても便利でしょう.

\begin{itemize}
\item \url{http://api.dartlang.org} Dart API Docsでは,すべてのライブラリについての機能,クラス,メソッドを掲載しています.
\item \url{http://www.dartlang.org/docs/language-tour/} The Dart Language TourはDart言語についてのハイレベルなガイドです.
\item \url{http://www.dartlang.org/docs/library-tour/} The Dart Library TourはDartプラットフォームにバンドルされているライブラリのほとんどをカバーしています.
\item \url{http://synonym.dartlang.org/} Javascriptを知っていますか?これはあなたのためのリソースです.Javascriptにおける一般的なイディオムやパターン,そして,スニペットをDartで書き換えたらどうなるかが書かれています.
\end{itemize}

\clearpage
