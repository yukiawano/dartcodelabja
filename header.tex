\documentclass[12pt]{jbook}

\usepackage{url}
\usepackage{ascmac}
\usepackage[dvipdfm]{graphicx}

\title{Dart Code Lab 日本語訳\footnote{オリジナルテキストhttp://www.dartlang.org/slides/2012/06/io12/Bullseye-Your-first-Dart-app-Codelab-GoogleIO2012.pdf}}
\author{GDG Kyoto 翻訳}
\date{最終更新日 \today}
\begin{document}
\begin{titlepage}
\maketitle
\thispagestyle{empty}
\end{titlepage}
\tableofcontents
\markboth{Dart CodeLab テキスト(日本語)}{Dart CodeLab テキスト(日本語)}

% Image Command
\newcommand{\img}[1]{\vspace{5mm}\includegraphics{#1}\vspace{5mm}}
\newcommand{\imgw}[1]{\vspace{5mm}\includegraphics[width=12.0cm]{#1}\vspace{5mm}}

% Original Style Chapter command
\newcommand{\ochapter}[1]{\chapter*{#1}
\addtocounter{chapter}{1}
\setcounter{section}{1}
\addcontentsline{toc}{chapter}{#1}}

% If you fail... message
\newcommand{\ifyougetstuck}[1]{もし,作業につまづいた場合は#1のフォルダの内容をstart-hereフォルダにコピーすることで,この部分の作業内容を終わらせた状態にすることができます.}

% boxes
\newcommand{\hint}[1]{
\vspace{5mm}
\begin{itembox}[l]{ヒント}
#1
\end{itembox}}

\newcommand{\advancedtopic}[1]{
\vspace{5mm}
\begin{itembox}[l]{発展的なトピック}
#1
\end{itembox}
}

\newcommand{\tip}[1]{
\vspace{5mm}
\begin{itembox}[l]{豆知識}
#1
\end{itembox}
}


\newpage

