\section{Step 8: JavaScriptにコンパイルし, 他のブラウザでも動作させよう}

Dartの機能の一つとして、JavaScriptへコンパイルしブラウザ上で動作させる機能があります。

\subsection{目的}

\begin{enumerate}
\item クライアントアプリをJavaScriptへコンパイルします。
\item 一般のブラウザでDartアプリケーションを実行します。
\item Dartiumブラウザと一般のブラウザでDartコードを実行する方法を学びます。
\end{enumerate}

\subsection{ウォークスルー}

finishedディレクトリを開きます。finishedディレクトリのコードを利用することで、アプリケーションをコンパイルして動作することを確かなものとします。start-hereから始めた場合でも、今までのステップをすべて済ませDartiumで動作しているなら大丈夫です。

finished/client/chat-client.dart ファイルを選択します。

[Image]

メニューからToolsを選択し、Generate JavaScriptを選択します。これは、dart2jsコンパイラを使うことで、DartクライアントアプリをJavaScriptに変換します。

[Image]

Dartエディタの下に表示されているconsoleをチェックし、コンパイルが成功しているか確認します。

[Image]

コンパイルに成功すると、finished/client/chat-client.dart.js ファイルが生成されます。

[Image]

chat-server.dart アプリケーションが動作していることを確認する方法は、前のstepを参照してください。

finished/client/chat-client.dart を実行し、クライアントアプリケーションを動作させます。すると、Dartiumが起動します。詳しい方法は前のstepを参照してください。

現在アプリケーションはDartiumで動作しています。URLを選択し、クリップボードにコピーします。

[Image]

Chromeを開き(Dariumではありません!)URLを貼り付けます。Dartiumとの間でチャットをし、JacaScriptに変換されたコードがChromeで動作していることを確認します。

[Image]

\begin{itemize}
\item {\bf Note:} 2012-06-26現在、Firefoxでは接続できません。これはDartの問題では無く、WebSocketの問題だと考えています。
\item {\bf Note:} 2012-06-26現在、Safariでは最新のWebSocketが実装されていないため動作しません。この件は次のページを参照してください。 \url{http://code.google.com/p/dart/issues/detail?id=3631}
\end{itemize}

URLは一つなのにDartiumとDartium以外のブラウザの両方で動作しています。どういう仕組みかは finished/client/index.html を開き、次の部分を確認してみましょう。

\begin{verbatim}
<script type="application/dart" src="chat-client.dart"></script>
<script src="dart.js"></script>
\end{verbatim}

dart.jsファイルはブラウザにDart VMの機能が含まれているかをチェックします。もし無ければapplication/dartスクリプトを削除し、text/javascriptスクリプトに置き換え、chat-client.dart.jsが動作するようにします。

\subsection{上級編}

DartでGMailをリビルドします。もしくは、勉強のためdart.jsを開きます。
時間があるのであれば、チャットルームに入っているメンバーのリストを取得するコマンドを追加してみてください。
または、チャットルーム参加時に、チャットルーム参加者のリストを送るように修正してみてください。


