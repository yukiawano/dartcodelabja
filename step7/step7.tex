\section{Step 7: Send and receive data with WebSockets}

WebSockets25 are part of the HTML5 family of technologies. They enable bi-directional full-duplex communication between a modern web browser and a WebSocket server. WebSockets can stream discrete messages, either text or binary data.

WebSockets are more efficient than long polling or constant querying over HTTP. They are ideal for chat applications! :)

\subsection{Objectives}

\begin{enumerate}
\item Learn about WebSockets
\item Connect to a WebSocket server
\item Receive data from a WebSocket
\item Send data to a WebSocket
\end{enumerate}

\subsection{Code}

If you need to catch up, you can copy step07 onto start-here for this portion of the codelab.

\subsection{Walkthrough}

Open client/chat-client.dart and find the ChatConnection class. You will now add a WebSocket connection, listen for messages, and send messages.

Tip: Open the API docs for the WebSocket class at http://api.dartlang.org/html/WebSocket.html.

\subsubsection{Connect to the WebSocket server}

Add a new instance field for the WebSocket to the ChatConnection class.

\begin{verbatim}
// client/chat-client.dart
class ChatConnection {
// Step 7. Add webSocket instance field.
WebSocket webSocket;
String url;
ChatConnection(this.url) {
\end{verbatim}

When the WebSocket instance is instantiated, it attempts to connect to the server at the URL. Find \_init inside of ChatConnection and add the following code.

\begin{verbatim}
// client/chat-client.dart
_init() {
// Step 7. Connect to the WebSocket, listen for events.
chatWindow.displayNotice("Connecting to Web socket");
webSocket = new WebSocket(url);
\end{verbatim}

First, the code displays a message to the chat window indicating that it's attempting a connection to the WebSocket server. Second, it instantiates a new WebSocket object, which attempts to connect to the WebSocket server.

Use the Open Declaration feature of the editor to determine where the url variable comes from.

[Image]

Notice how the url variable is an instance variable of ChatConnection, originally set in the constructor.

\subsubsection{Handle WebSocket events}

WebSocket objects respond to four events: open, close, error, and message. The message event is fired when a new message is received on the WebSocket. The content of the message is found in event.data.

Tip: Open the API docs for the WebSocket events at http://api.dartlang.org/html/
WebSocketEvents.html.

Handle the four WebSocket events inside of ChatConnection's \_init.

\begin{verbatim}
// client/chat-client.dart
_init() {
// Step 7. Connect to the WebSocket, listen for events.
chatWindow.displayNotice("Connecting to Web socket");
webSocket = new WebSocket(url);
webSocket.on.open.add((e) {
chatWindow.displayNotice('Connected');
});
webSocket.on.close.add((e) {
chatWindow.displayNotice('web socket closed');
});
webSocket.on.error.add((e) {
chatWindow.displayNotice("Error connecting to ws");
});
webSocket.on.message.add((e) {
print('received message ${e.data}');
});
}
\end{verbatim}

Now that the event listeners for WebSocket events are configured, you are ready to use the \_receivedEncodedMessage() method you created from Step 5.

\begin{verbatim}
// client/chat-client.dart
webSocket.on.message.add((e) {
print('received message ${e.data}');
_receivedEncodedMessage(e.data);
});
\end{verbatim}

\subsubsection{Send messages to the WebSocket server}

Finally, with the webSocket instance connected, you can send messages to the server. Find \_sendEncodedMessage() in the ChatConnection class and send the encoded message to the WebSocket server. Only send the message if the webSocket object exists and is actually connected to the server.

\begin{verbatim}
// client/chat-client.dart
_sendEncodedMessage(String encodedMessage) {
// Step 7. Send the message over the WebSocket.
if (webSocket != null && webSocket.readyState == WebSocket.OPEN) {
webSocket.send(encodedMessage);
} else {
print('WebSocket not connected, message $encodedMessage not sent');
}
}
\end{verbatim}

\subsubsection{Test the app}

Everything should be wired up, time to test your app!

Run the chat-server.dart, if not already running (see Step 2 for instructions). Next, run the client/chat-client.dart application, which launches Dartium. The chat window will print [system]: Connected if the client connects to the WebSocket server.

[Image]

Check the chat-server.dart output in the Dart Editor to see "new ws conn message", which confirms your client has connected to the WebSocket server.

[Image]

Not connecting to the server? Make sure the chat-server.dart is running. Sometimes, completely shutting down and restarting Dartium helps.
Send some messages! Copy the URL in Dartium, open a new Dartium tab, and paste the URL in. Enter a different username, and start chatting between the two windows.

[Image]

\subsection{Advanced}

Handle connection failures by retrying the connection attempt. Add increasing back-off retry delays.

Update the \_init() method to take a number of seconds to wait before retrying the connection. Add a boolean to track if an error was encountered (you will use this shortly).

\begin{verbatim}
// client/chat-client.dart
_init([int retrySeconds = 2]) {
// Step 6
bool encounteredError = false;
\end{verbatim}

In both the close and error event handlers, add a retry to reconnect after a delay. Set the encounteredError boolean to true so the retry timer is scheduled only once, in the event that both close and error events are fired.

\begin{verbatim}
// client/chat-client.dart
webSocket.on.close.add((e) {
chatWindow.displayNotice('web socket closed, retrying in $retrySeconds
seconds');
if (!encounteredError) {
window.setTimeout(() => _init(retrySeconds*2), 1000*retrySeconds);
}
encounteredError = true;
});
webSocket.on.error.add((e) {
chatWindow.displayNotice("Error connecting to ws");
if (!encounteredError) {
window.setTimeout(() => _init(retrySeconds*2), 1000*retrySeconds);
}
encounteredError = true;
});
\end{verbatim}

The setTimeout() method takes two arguments: a function and when it should be called (in milliseconds from now). In this case, the \_init() method runs after retrySeconds. This is a good example of Dart’s lexical closures, because the inline function for setTimeout() easily references both \_init() and retrySeconds.

